%
% Copyright (c) 2017  Pavel Kirienko <pavel.kirienko@zubax.com>
%

\documentclass{zubaxdoc}
\graphicspath{{document_templates/documentation_template_latex/}}

\usepackage{xstring}
\usepackage{multirow}
\usepackage{diagbox}
\usepackage{amsmath}
\usepackage{ConfigParamIndex}

% Draft watermark
%\usepackage{draftwatermark}
%\SetWatermarkLightness{0.85}
%\SetWatermarkText{DRAFT}

\hbadness=10000

\title{Sapog v3 Reference Manual}

\begin{document}
\frontmatter

\begin{titlepage}
	\section*{Overview}\label{sec:overview}
	
	Sapog is an advanced open source sensorless PMSM/\allowbreak{}BLDC motor controller firmware designed for
	use in propulsion systems of electric unmanned aircraft and watercraft.
	
	The source repository and the public bug tracker are located at
	\url{https://github.com/PX4/sapog}.
	
	This document is applicable to firmware versions 2.x released before \today.
	
	This document focuses only on the firmware.
	Please refer to your product documentation for relevant information about the hardware.
	
	\section*{Applications}
	\begin{itemize}
		\item Propeller drives of unmanned aerial vehicles.
		\item Watercraft propulsion systems.
		\item General purpose sensorless BLDC drives.
	\end{itemize}
	
	\BeginRightColumn
	\section*{Features}
	\begin{itemize}
		\item Robust motor control algorithms.
		\item Fast response (critical for multirotor aircraft).
		\item Regenerative braking and active freewheeling.
		\item Optional RPM control loop (RPM governor).
		\item Current limiting.
		\item Self diagnostics and extensive real-time status reporting.
		\item Compatible with most BLDC motors with minimal tuning.
		\item Highly configurable.
		\item Automatic firmware update over the CAN bus in the field.
		\item Supported communication interfaces:
		\begin{itemize}
			\item UAVCAN interface with optional dual bus redundancy.
			\item Command line interface over UART, suitable for M2M communications.
			\item RCPWM (analog PWM interface widely used in robotics).
		\end{itemize}
	\end{itemize}
\end{titlepage}

\tableofcontents
\clearpage
\listoffigures
\BeginRightColumn
\listoftables

\mainmatter

\chapter{Introduction}

\section{Overview}

This document provides detailed information about the basic operating principles and
implementation details of the Sapog BLDC motor controller firmware.

This document is focused exclusively on the aspects pertaining to the firmware itself.
All questions related to particular hardware implementations are intentionally left out;
that information should be gathered from the hardware datasheets distributed by
hardware vendors.

\section{Conventions}

Unless specifically stated otherwise, all units of measurement used in this document are assumed to be SI.
Units of measurement are typically shown in square brackets: $\left[\text{unit}\right]$.

Within equations, references to configuration parameter values are made through the symbol $\Pi$
with the name of the configuration parameter in the subscript;
for example: $\Pi_\text{parametername}$.

\section{Definitions}

\begin{description}
	\item[PMSM] Permanent magnet synchronous motor.
	\item[BLDC] Brushless DC motor.
	\item[RPM] Revolutions per minute.
	\item[EMF] Electromotive force.
	\item[BEMF] Back electromotive force.
	\item[ADC] Analog to digital converter.
	\item[VSI] Voltage source inverter.
	\item[ESC] Electronic speed controller, synonymous to PMSM controller in the context of electric vehicles.
\end{description}

\chapter{Principles of operation}

This chapter provides a brief overview of the basic operating principles of electric permanent magnet
synchronous motors and of the relevant approaches and solutions implemented in Sapog.

\section{PMSM and BLDC motors}

\subsection{Basic equations}\label{sec:motor_equations}

This section provides some of the most important equations that describe a DC electric motor.
The principles explained here can be extended to all types of polyphase PMSM motors as well.

In the context of electric drives, BEMF is the voltage induced on the motor windings
while the armature of the motor is moving relative to the magnetic field of the rotor.
The magnitude of the induced voltage is dependent on the speed of the armature relative to the magnetic field
of the rotor and on the magnetic flux linkage of the motor.
The dependency can be expressed as follows:
\begin{equation}
	E_b = \phi \omega_e
\end{equation}
where $E_b$ $\left[\text{volt}\right]$ is the induced back electromotive force,
$\phi$ $\left[\text{weber}\right]$ is the magnetic flux linkage of the motor,
and $\omega_e$ $\left[\frac{\text{radian}}{\text{second}}\right]$ is the electrical angular velocity.

Electrical angular velocity is a function of the mechanical angular velocity
$\omega_m$ $\left[\frac{\text{radian}}{\text{second}}\right]$
and the number of rotor magnetic poles $N_p$:
\begin{equation}\label{eq:speed_electrical_mechanical}
	\omega_e = \frac{\omega_m N_p}{2}\qquad
	N_p \geq 2, N_p\text{\ is even}
\end{equation}

Both mechanical and electrical angular velocities are related to the mechanical and electrical RPM,
respectively, as follows:
\begin{equation}
	\text{RPM} = \frac{30 \omega }{\pi }
\end{equation}

The armature current $I_a$ $\left[\text{ampere}\right]$ is proportional to
the voltage difference between the induced back EMF and
the source voltage $E_s$ $\left[\text{volt}\right]$:
\begin{equation}
	I_a = \frac{E_s - E_b}{R}
\end{equation}
where $R$ $\left[\text{ohm}\right]$ is the internal resistance of the stator.

Torque $\tau$ $\left[\text{newton}\cdot{}\text{meter}\right]$ observed on the shaft is dependent on the
torque-generating current, magnetic flux linkage, and the number of rotor magnetic poles as follows:
\begin{equation}
	\tau = \frac{\sqrt{3}}{2} I_a N_p \phi
\end{equation}

Mechanical power output $P$ $\left[\text{watt}\right]$
is the product of torque and mechanical angular velocity:
\begin{equation}
	P = \tau \omega_m
\end{equation}

Motor velocity constant $K_V$ $\left[\frac{\text{RPM}_m}{\text{volt}}\right]$
is sometimes used in marketing materials and motor specifications
instead of magnetic flux linkage.\footnote{Sometimes $K_V$ is written as KV, although
	this form is discouraged because it can be confused with kilovolt (kV).}
This is not a SI quantity, so its use is generally discouraged.
It can be expressed via the magnetic flux linkage $\phi$
and the number of rotor magnetic poles $N_p$ as follows:
\begin{equation}
	K_V = \frac{20 \sqrt{3}}{\pi  N_p \phi }
\end{equation}

\subsection{BEMF shape}

There are two major types of permanent magnet synchronous motors (PMSM) that differ by the shape of the induced
back electromotive force while the rotor is moving at a constant speed.

The first type has a fixed magnetic flux linkage that is independent of the electrical position of the rotor.
The induced BEMF per phase therefore has a sinusoidal shape.
These motors are typically referred to simply as PMSM, or BLAC.
Since the term PMSM is quite ambiguous, we will be using the term BLAC to refer to a PMSM with sine shaped BEMF.

The other type, referred to as brushless DC, or BLDC, has a variable flux linkage that changes with the
rotor position in a specific way that results in the trapezoidal shape of the back EMF.
The difference is illustrated on figure~\ref{bemf_trapezoidal_vs_sinusoidal}.

\begin{figure}[hbt]
	\centering
	\includegraphics[width=0.6\textwidth]{bemf_trapezoidal_vs_sinusoidal}
	\caption{Difference between trapezoidal and sinusoidal BEMF
		\label{bemf_trapezoidal_vs_sinusoidal}}
\end{figure}

Sapog, being a BLDC drive, drives the motor with trapezoidal modulated voltage.
This produces smooth torque and vibration free operation with BLDC motors,
since it is expected that the flux linkage will be changing in a specific way
that matches the shape of the modulated voltage.

Like any BLDC drive, Sapog can operate with BLAC motors as well, albeit with increased torque ripple.
From the model presented in section~\ref{sec:motor_equations} we can predict that the motor
will exhibit periodic variations of the output torque if the form of the voltage modulated by the controller
does not match the form of the back EMF induced in the motor.
The resulting periodic torque variations that occur in a BLAC motor driven by a BLDC drive
(and, likewise, in a BLDC motor driven with sine modulated voltages) are shown on the figure
\ref{sine_torque_deviation}.
Many applications, however, can tolerate the resulting torque ripple and vibrations.

\begin{figure}[hbt]
	\centering
	\includegraphics[width=0.6\textwidth]{sine_torque_deviation}
	\caption{Torque ripple caused by mismatching forms of the back EMF and the source voltage
		\label{sine_torque_deviation}}
\end{figure}

\section{Motor control}

\subsection{Commutation sequence}

\newcommand{\BEMFH}{$\uparrow$}
\newcommand{\BEMFL}{$\downarrow$}

Three phase BLDC drives operate by modulating a specific sequence of voltages on the motor phases
synchronously with the movement of the rotor.
The modulated voltage sequences are shown in the commutation table below
(legend: $+$ -- positive voltage output; $-$ -- negative voltage output;
\BEMFH{} -- the phase is floating, induced BEMF is rising;
\BEMFL{} -- the phase is floating, induced BEMF is falling).

\begin{tabu}{|l c|c c c c c c|}
	\hline
	\rowfont{\bfseries}
	\multicolumn{2}{|c|}{\diagbox{Phase}{Step}}
	& 0     & 1     & 2     & 3     & 4     & 5     \\\hline
	\multirow{3}{*}{Forward} & A & $-$   &\BEMFH & $+$   & $+$   &\BEMFL & $-$   \\
	& B & $+$   & $+$   &\BEMFL & $-$   & $-$   &\BEMFH \\
	& C &\BEMFL & $-$   & $-$   &\BEMFH & $+$   & $+$   \\\hline
	\multirow{3}{*}{Reverse} & A & $-$   &\BEMFH & $+$   & $+$   &\BEMFL & $-$   \\
	& B &\BEMFL & $-$   & $-$   &\BEMFH & $+$   & $+$   \\
	& C & $+$   & $+$   &\BEMFL & $-$   & $-$   &\BEMFH \\\hline
\end{tabu}

The choice between forward and reverse commutation tables affects the direction of rotation of the motor,
and it is specified via the configuration parameter \CfgRef{ctl+dir}.

The rotor position is deduced from the behavior of the induced BEMF on the floating phase.
From figure~\ref{bemf_trapezoidal_vs_sinusoidal} we can deduce that when the induced BEMF of the floating
phase crosses the median voltage between the high and the low phase, there are 30 electrical degrees left
before the next commutation step begins.
The controller leverages this rule to estimate the time when the next commutation should occur.
The process is demonstrated on figure~\ref{commutation_basics}.
The voltage waveforms shown on the figure were acquired from the phase voltage signal conditioning circuits,
before the ADC inputs (as shown on figure~\ref{power_stage_schematic}).

\begin{figure}[hbt]
	\centering
	\includegraphics[width=\textwidth]{commutation_basics}
	\caption{Six step commutation sequence observed through phase voltages
		\label{commutation_basics}}
	PWM modulation is not seen because the shown voltages were acquired while the controller
	was operating at the full duty cycle.
\end{figure}

\begin{figure}[hbt]
	\centering
	\includegraphics[width=0.8\textwidth]{power_stage_schematic}
	\caption{Simplified schematic of the VSI and BEMF feedback circuits that Sapog relies on
		\label{power_stage_schematic}}
\end{figure}

From section~\ref{sec:motor_equations} we know that the magnitude of the induced BEMF is linearly
dependent on the speed of the relative motion between the armature and the rotor's magnetic field.
Considering also the fact that the controller relies on the induced BEMF signal for rotor position estimation,
we can predict that the rotor position estimation may be unreliable if the induced BEMF is not sufficiently
strong.
In order to avoid issues with low speed operation, Sapog provides several configuration parameters that
restrict the minimum operating speed, such as the following:

\begin{itemize}
	\item \CfgRef{mot+comm+per+max} - maximum commutation period, in microseconds.
	The estimated commutation period will be constrained by the controller to not exceed this value.
	\item \CfgRef{mot+rpm+min} - minimum RPM that can be demanded in the RPM governor mode.
	If the demanded RPM is lower, the setpoint will be increased by the controller.
	Reduce this parameter to make the minimum speed lower; increase it to improve stability.
	\item \CfgRef{mot+v+min} - minimum source voltage when operating in the open loop mode, in volts.
	If the demanded duty cycle amounts to a lower source voltage than this, the setpoint will be
	increased by the controller.
	Reduce this parameter to make the minimum speed lower; increase it to improve stability.
\end{itemize}

These parameters may need adjustment to make Sapog compatible with a specific motor.
For example, if the motor cannot maintain stable operation at a low speed,
the minimum source voltage should be increased,
and/or the maximum commutation period should be made longer.
The minimum voltage limit does not apply to the RPM loop mode,
where the minimum speed is governed by a dedicated configuration parameter as shown above.

The following equation can be used to convert between commutation period $T_\text{comm}$
$\left[\text{second}\right]$
and electrical angular velocity $\omega_e$ $\left[\frac{\text{radian}}{\text{second}}\right]$:
\begin{equation}
	\omega_e = \frac{\pi}{3 T_\text{comm}}
\end{equation}

\subsection{Source voltage modulation}

The source voltage modulation technique employed by Sapog is a fixed carrier frequency, fixed dead time,
4-quadrant, simultaneous complementary switching PWM modulation.
This modulation technique supports active freewheeling and regenerative braking.

Reviewing different modulation techniques would be outside of the scope of this document.
For more background, the reader is advised to refer to other sources, such as the article
``Influence of PWM Schemes and Commutation Methods for DC and Brushless Motors and Drives''
(Dal Y. Ohm, Richard J. Oleksuk).

The following configuration parameters govern the operation of the source voltage modulator:
\begin{itemize}
	\item \CfgRef{mot+pwm+hz} - PWM carrier frequency, in hertz.
	\item \CfgRef{mot+pwm+dt+ns} - PWM switching dead time, in nanoseconds.
\end{itemize}

\subsubsection{PWM dead time considerations}

Generally, the end user should not change the dead time settings from the default values provided by
the vendor, because that may have unexpected negative effects on the drive.

However, in certain scenarios where a higher power output is preferred, the application may benefit from
reduced dead time intervals.
The hardware documentation may contain more useful data on this topic.

\subsection{Field weakening}\label{sec:field_weakening}

From section~\ref{sec:motor_equations} we know that the maximum speed that a motor can achieve is dependent
on the the load, magnetic flux linkage, and the source voltage.
Reduction of the magnetic flux linkage decreases the induced BEMF, which allows the rotor to reach a higher speed,
given a constant source voltage and load.

The controller can alter the commutation sequence in a specific way, causing the rotating magnetic field
created by the stator to interact with the magnetic field of the rotor in a way that reduces the magnetic
field linkage, effectively increasing the maximum speed of the rotor, reducing the output torque,
while keeping the maximum mechanical power output constant.

This technique is known as \emph{field weakening}, and it is implemented by means of adjustment of the
commutation sequence, shortening the time interval between the point of zero crossing and the moment when the next
commutation step begins.
In the context of BLDC drives this specific approach is also known as \emph{commutation advancing}.

Observe that since a part of the input power is used for the suppression of the magnetic fields induced by the
permanent magnets in the rotor, the overall efficiency of the drive decreases with stronger field weakening.

Sapog defines the commutation advance angle in electrical angular degrees of rotor position,
and it can vary from 0\degree{} to 29\degree{}, inclusively, where 0\degree{} corresponds to no advance,
and 29\degree{} corresponds to the maximum advance angle.
The practical effects imposed on the drive are summarized in the following table.

\begin{ZubaxCompactTable}{|c|c c c c|}
	Advance angle & Flux linkage & Maximum torque & Maximum speed & Drive efficiency \\
	Low           & High         & High           & Low           & High             \\
	High          & Low          & Low            & High          & Low              \\
\end{ZubaxCompactTable}

Sapog can modulate a fixed advance angle, or it can be defined as a function of speed.
The latter is especially useful, because a strong field weakening at low speeds does not make practical sense,
and serves only to reduce the drive efficiency with no apparent benefit for the application.
Additionally, a strong field weakening may be responsible for stability issues at low operating speeds,
because it effectively reduces the magnitude of the BEMF induced in the motor.

The field weakening feature is configured with the help of the following configuration parameters:

\begin{itemize}
	\item \CfgRef{mot+tim+adv+max} - the maximum commutation advance angle, in electrical rotor position degrees.
	\item \CfgRef{mot+tim+adv+min} - the minimum commutation advance angle, same units.
	\item \CfgRef{mot+tim+cp+min} - when the current commutation period is longer than this value,
	the minimum advance angle will be used. The units are microseconds.
	\item \CfgRef{mot+tim+cp+max} - when the current commutation period is shorter than this value,
	the maximum advance angle will be used. Same units.
\end{itemize}

When the current commutation period is between \verb|mot_tim_cp_max| and \verb|mot_tim_cp_min|,
the advance angle will be proportionally linearly interpolated between
\verb|mot_tim_adv_min| and \verb|mot_tim_adv_max|.
The interpolation logic is demonstrated on figure~\ref{timing_advance_interpolation_plot}.

\begin{figure}[hbt]
	\centering
	\includegraphics[width=0.7\textwidth]{timing_advance_interpolation_plot}
	\caption{Timing advance interpolation logic
		\label{timing_advance_interpolation_plot}}
\end{figure}

Sample phase voltage waveforms acquired while the controller was operating with high field
weakening settings are shown on figure~\ref{phase_voltages_at_high_advance_angle}.

\begin{figure}[hbtp]
	\centering
	\includegraphics[width=\textwidth]{phase_voltages_at_high_advance_angle}
	\caption{Commutation sequence with high advance angle settings
		\label{phase_voltages_at_high_advance_angle}}
	It can be seen that the next commutation begins shortly after the induced BEMF
	crosses the neutral voltage.
\end{figure}

\subsubsection{Freewheeling considerations}

Upon careful review of the field weakening principle it might become obvious that it poses a certain danger
when the motor operates at a high speed.

According to the explanations above, as long as the field weakening is active,
the controller actively suppresses the excessive flux of the permanent magnets in the rotor.
The suppression allows the controller to lower the induced BEMF, and, therefore,
extend the operating speed upwards, while keeping the source voltage constant.

Consider what happens when the inverter (or the controller as a whole) becomes turned off while the
motor is running with the rotor field weakened.
As the controller ceases to suppress the field, the flux linkage of the motor returns to nominal,
which, given that the speed is sufficiently high, may significantly exceed the inverter supply
voltage $V_\text{inv}$, possibly causing damage to the controller, the power supply, the motor, and so on.

\subsection{Rotor state observer}

\subsubsection{BEMF processing}

As mentioned previously, Sapog deduces the current position of the rotor by tracking the induced BEMF
of the floating phase.
Which phase is floating and the direction in which the BEMF is changing is dependent on the current electrical
angle of the rotor.

Measurement of the induced BEMF is done by the ADC, which is connected to the motor phases via
resistive voltage dividers which reduce the dynamic range of the motor phase voltages down to a narrower
dynamic range acceptable for the ADC,
and a buffer capacitor that provides fast recharge of the sample-and-hold capacitor of the ADC,
and suppresses high frequency noise.
The resulting simplified schematic can be seen on figure~\ref{power_stage_schematic}.

BEMF measurements are sensitive to noise.
In order to prevent the noise emitted by the VSI from interfering with BEMF measurements,
the controller synchronizes the BEMF sampling with PWM switching.
However, the temporal resolution provided by the resulting sampling pattern is insufficient for robust
rotor state estimation because of the sampling period restrictions imposed by the
PWM carrier frequency.

In order to work around this limitation, Sapog collects a number of samples, and,
knowing that the shape of the BEMF induced by the floating phase of a BLDC motor is linear,
solves a linear regression problem on the collected samples, effectively recovering the most probable
shape of the BEMF signal.
The resulting approximation is very robust and resilient to noise.

The principle is illustrated on figure~\ref{phase_voltage_sampling}.

\begin{figure}[hbtp]
	\centering
	\includegraphics[width=\textwidth]{phase_voltage_sampling}
	\caption{BEMF sampling and PWM modulation
		\label{phase_voltage_sampling}}
\end{figure}

The number of BEMF samples $N_\text{samples}$ considered in the linear regression problem is defined by
the configuration parameter \CfgRef{mot+bemf+win+den} (dimensionless) as follows:
\begin{equation}
	N_\text{samples} =
	\frac{T_{\text{comm}} F_{\text{pwm}}}{\frac{\alpha\cdot{}\Pi_\text{bemfwinden}}{15} + \Pi_\text{bemfwinden}} + 2
\end{equation}
where $T_{\text{comm}}$ is the current commutation period,
$\Pi_\text{bemfwinden}$ is the value of the configuration parameter,
$\alpha$ is the current commutation advance angle in electrical rotor position degrees,
and $F_{\text{pwm}}$ $\left[\text{hertz}\right]$ is the PWM carrier frequency.

An attentive reader might point out that the assumption of linear BEMF that the described method rests
upon does not hold for BLAC motors, since the induced BEMF of a floating phase will be sinusoidal rather
than linear.
While the objection is correct, this deviation does not pose a significant problem, since the sinusoidal
BEMF can be effectively approximated to linear near the point of zero crossing.
Some issues may arise with very high advance angle settings, where the controller may be forced to resort to
the look-ahead extrapolation, so that the linear regression will be applied to a sample set acquired
at the point where the linear approximation is not valid.
The resulting imprecision, however, should not be of a significant practical importance.

\subsubsection{Special cases}

Figure~\ref{phase_voltage_sampling_corner_cases} provides a closer look at the challenges
seen by the controller in certain operating modes, which are reviewed below.

\begin{figure}[hbtp]
	\centering
	\includegraphics[width=\textwidth]{phase_voltages_braking_high_advance_angle}
	\caption{Corner cases of BEMF sampling
		\label{phase_voltage_sampling_corner_cases}}
\end{figure}

In certain operating modes the induced BEMF of the floating phase may briefly exceed the
inverter supply voltage $V_\text{inv}$, or fall below the ground level,
causing a voltage drop on either the high-side or low-side flyback diodes, respectively.
The BEMF samples acquired at this time are not representative of the true form of the signal,
because the signal is altered by external factors;
and therefore, such samples cannot be used for the linear regression.
Sapog uses a simple heuristic that allows it to detect when the sampled signal is subjected to
altering and discard the affected samples.
The heuristic checks whether the sampled voltage falls into a specified range near the neutral voltage,
and discards the sample if it doesn't.

Additionally, the samples acquired shortly after a commutation may be subjected to transient distortions.
Sapog works around that by imposing a fixed \emph{blanking time} after every commutation, during which
the BEMF is not sampled.

The following configuration parameters are relevant to the reviewed concepts:

\begin{itemize}
	\item \CfgRef{mot+bemf+range} - the voltage range around the neutral,
	in percent of the inverter supply voltage $V_\text{inv}$, where the BEMF samples
	will be considered valid and accepted for the regression problem.
	\item \CfgRef{mot+blank+usec} - the duration of the blanking time after the commutation, in microseconds.
	\item \CfgRef{mot+spup+blnk+pm} - the minimum blanking time during spin-up,
	permill of the current commutation period.
	This parameter is reviewed in more detail in a dedicated section.
\end{itemize}

Additional difficulties are posed by operation at very high power levels,
and/or when the controlled motor has highly inductive windings.
Under these conditions, the motor windings tend to accumulate a significant energy during the duration
of the commutation period.
When the current commutation period ends and the winding becomes floating,
the accumulated energy maintains some current through the winding and through the \emph{flyback diode}
of one of the transistors until the energy accumulated while the winding was powered is dissipated.
The current caused by the energy release from a disconnected inductive network is known as
\emph{flyback current}.
As long as the flyback current is flowing though the winding and the flyback diode,
the induced BEMF is completely suppressed and, therefore,
is unobservable, which does not allow the controller to evaluate the state of the rotor.

\begin{figure}[hbtp]
	\centering
	\includegraphics[width=\textwidth]{phase_voltages_at_high_load}
	\caption{Challenges of high power operation
		\label{phase_voltages_at_high_load}}
\end{figure}

Sapog recognizes this condition and waits for the flyback current to cease.
If the current is still flowing past a certain deadline necessary for the controller to evaluate the
point of zero crossing, Sapog enters a \emph{desaturation mode} in order to bring the system
back into an observable state.
In the desaturation mode Sapog turns off the inverter and waits until the next commutation period begins.
By the time the next commutation period begins, the flyback currents will have ended,
allowing the controller to continue observation of the motor state.

Figure~\ref{phase_voltages_at_high_load} shows operation waveforms near the point of activation
of the desaturation mode.

\subsubsection{Maximum speed restrictions}

It is easy to see that as the speed increases and the commutation period shortens,
the number of BEMF samples that can be accommodated in a commutation period decreases.
This effect imposes a limit on the maximum achievable speed,
and the limit is dependent on the PWM carrier frequency.

When the drive operates in the open loop mode, the controller will limit the maximum duty cycle
when the commutation period reaches dangerously low values.
The limiting is done by means of a simple P-controller,
which overrides the duty cycle setpoint when it exceeds a certain value.

When the drive operates in the RPM control loop mode, the limiting will be performed simply by means of
clipping of the RPM setpoint.

The exact thresholds are specified on figure~\ref{pwm_maximum_speed}.

The configuration parameter \CfgRef{mot+pwm+hz} can be used to set the desirable PWM frequency
in hertz.
It is recommended to keep the frequency as low as possible,
as that would facilitate lower switching losses and therefore higher efficiency of the drive.

\begin{figure}[hbt]
	\centering
	\includegraphics[width=\textwidth]{pwm_maximum_speed}
	\caption{Dependency of the maximum speed on PWM frequency
		\label{pwm_maximum_speed}}
\end{figure}

\subsubsection{Field weakening restrictions}

The controller requires several microseconds to solve the linear regression problem after the sampling is
finished.
However, operation at very high advance angles requires the controller to perform commutation virtually
immediately after the zero crossing has happened, leaving very little time to perform the computation.
The controller strives to work around this contradiction by trying to robustly extrapolate the shape of the
BEMF signal, predicting the upcoming zero crossing ahead of time.

Despite that, operation at very high speeds may not provide the time necessary to take a sufficient
number of samples and solve the regression problem before the next commutation is due.
In this case, Sapog will silently lower the current advance angle setting and keep it at as close
to the requested value as the current operating mode permits.

\subsection{Regenerative braking}

During a regenerative braking, the source voltage $E_s$ modulated by the controller is lower than the
induced BEMF $E_b$, which creates a reverse current flow from the motor into the VSI
and further into the power supply network.
According to the formulas presented in section~\ref{sec:motor_equations},
reversing the current will also reverse the torque, causing the motor to rapidly decelerate.

This process converts the mechanical energy of the motor into electrical energy that is returned back into the
power supply network.
If the power supply network cannot absorb the recovered energy (e.g. due to its resistance being too high),
the regenerative energy transfer may lead to an increase of the supply voltage beyond the safe operating limits.
Typically, the controller's hardware is equipped with some protection circuits that activate and dissipate the
excessive energy when $V_\text{inv}$ increases a certain threshold;
however, their energy absorption capabilities are often very limited.

Generally, batteries are capable of absorbing the regenerated energy during braking without issues.
Problems may arise if the controller is powered from a source that does not permit reverse currents,
such as laboratory power supplies or some types of voltage conversion circuits.

Refer to the documentation provided with your hardware to learn more about the issues
concerning the regenerative braking and the excess energy release.

\subsection{Rotor stall detection}\label{sec:stall_detection}

Rotor stall is a condition where the rotor ceases to rotate due to the torque generated by the motor
being lower than the mechanical load on the shaft.
Sapog detects this condition by observing a specific number of failures to find
a valid BEMF zero crossing event.

As explained in one of the earlier sections of this document,
Sapog requires the rotor to move in order to be able to operate normally
(this applies to virtually all sensorless drives).
Once the rotor has stalled, Sapog shuts down the VSI and waits for the next setpoint to arrive.
Once the next setpoint has arrived, Sapog restarts the motor normally,
unless the number of consecutive stalls has exceeded a specific limit.

In the latter case, Sapog will lock up and refuse to start the motor again,
until it has received a zero setpoint.
A zero setpoint is treated as a reset command that clears the stall counter and unlocks the controller.

Zero crossing detection failures are counted with the help of a dedicated counter state.
When Sapog fails to find a zero crossing solution,
the failure counter is incremented by one.
When Sapog succeeds to perform six (6) commutations in a row without failures,
the failure counter is reset back to zero.

The following configuration parameters configure the behaviors pertaining to stall detection:

\begin{itemize}
	\item \CfgRef{mot+zc+fails+max} - zero crossing detection failure threshold.
	If the failure counter exceeds this threshold, the rotor will be considered stalled.
	\item \CfgRef{mot+stop+thres} - lock-up threshold.
	If the rotor has stalled this number of times in a row,
	the controller will lock up until a zero setpoint is received.
\end{itemize}

\subsection{Spin-up}

As has been explained before, Sapog requires the rotor to move in order to be able to operate normally,
because the state estimation methods that work well when the rotor is moving break apart when the rotor is
stationary or nearly so.

From the above follows that a different method of state estimation is needed to start the motor
from standstill, especially so if the shaft is attached to a mechanical load.

\subsubsection{Algorithm}

When starting the motor, Sapog sets the source voltage $E_s$ to the value specified in the
configuration parameter \CfgRef{mot+v+spinup}, and activates the commutation step at the index 1.
Having done so, Sapog begins to wait for the zero crossing detector to report that it is time to
begin the next commutation period, or until the amount of time specified by the configuration parameter
\CfgRef{mot+spup+st+cp} expires, whichever happens first.
Afterwards, Sapog switches to the next commutation step, and the process repeats again.

While in the spin-up mode, Sapog slowly increases the voltage from the initial value specified
by \CfgRef{mot+v+spinup} until the minimal operating voltage \CfgRef{mot+v+min} is reached.
The duration of the spin-up voltage ramp is specified by the parameter \CfgRef{mot+spup+vramp+t},
in seconds.
The ramp is graphically illustrated on figure~\ref{spinup_voltage_ramp}.

\begin{figure}[hbt]
	\centering
	\includegraphics[width=0.7\textwidth]{spinup_voltage_ramp}
	\caption{Voltage ramp during spin-up
		\label{spinup_voltage_ramp}}
\end{figure}

It should be noted that high initial voltages and/or steep voltage ramps need to be avoided,
because they may lead to severe instabilities in the beginning of the spin-up sequence.
Keep in mind that it is virtually impossible for any synchronous sensorless drive to create a stable
torque at low speeds, and especially at standstill.

While spin-up is in progress, Sapog always operates at the zero advance angle
(field weakening disabled) for the reasons explained earlier,
regardless of the configured advance angle settings.

\subsubsection{Spin-up rotor state observer}

The zero crossing detector mentioned above operates in a very different way compared to the
normal rotor state observer used in the normal operating mode.

While operating in the spin-up mode, the observable behavior of the induced BEMF may be very erratic,
mainly due to the low magnitude of the signal and also due to the irregular rotation of the rotor caused by
imprecise synchronization of the modulated field with the field of the rotor.
In order to combat these difficulties, Sapog employs a different kind of observer, which provides
more robust operation at low speeds.

Immediately after a commutation Sapog inserts a blanking delay $T_\text{blank}$
defined by the following equation:
\begin{equation}
	T_{\text{blank}}=
	\max \left(\frac{\Pi_{\text{blankusec}}}{10^6},
	\frac{\Pi_{\text{spupblnkpm}} T_{\text{comm}}}{1000}\right)
\end{equation}
where $\Pi_{\text{spupblnkpm}}$ is the value of the configuration parameter \CfgRef{mot+spup+blnk+pm}
(permill of the commutation period),
$T_{\text{comm}}$ is the duration of the last commutation period,
$\Pi_{\text{blankusec}}$ is the value of the configuration parameter \CfgRef{mot+blank+usec}
(in microseconds).

Upon expiration of the blanking delay, Sapog watches the BEMF waiting for the flyback current to cease.
Afterwards, Sapog begins to integrate the sampled BEMF voltages relative to the neutral voltage.
Once the integrated voltage changes the sign, Sapog switches to the next commutation period.
A more formal definition of the commutation condition is shown below:
\begin{equation}
	\begin{aligned}
		&\sum_{n} E_{s_n} - E_{\text{neutral}_n} > 0 \qquad\text{for rising BEMF}\\
		&\sum_{n} E_{s_n} - E_{\text{neutral}_n} < 0 \qquad\text{for falling BEMF}
	\end{aligned}
\end{equation}

The principle is illustrated on the oscillogram~\ref{spinup_phase_voltages}.

\begin{figure}[hbt]
	\centering
	\includegraphics[width=\textwidth]{spinup_phase_voltages}
	\caption{Spin-up sequence with a highly inertial load
		\label{spinup_phase_voltages}}
\end{figure}

\subsubsection{Termination conditions}

Spin-up will be considered finished once the maximum commutation period defined by the parameter
\CfgRef{mot+comm+per+max} \emph{and} the minimum voltage \CfgRef{mot+v+min} are reached.
If the target commutation period is reached sooner than the voltage ramp comes to the final
voltage level, the controller will significantly accelerate the ramp in order to reach
the minimum operating voltage faster.
Once the spin-up process is finished,
the controller will abandon the spin-up mode and switch into the normal mode.

If the target commutation period could not be reached in the amount of time specified by
the parameter \CfgRef{mot+spup+to+ms}, in milliseconds, the controller will abort the spin-up process,
shut down the VSI and register a regular rotor stall event.

\subsubsection{Spin-up from a non-stationary state}

Sapog assumes that the rotor is (nearly) stationary in the beginning of the spin-up process.
If this assumption is violated, there may be brief transient processes involving abnormal currents,
high vibrations and high EMI.

The transient currents can be either direct or regenerative, depending on whether the desired
and the actual directions of rotation match, and whether the initial guess of the current
commutation step was correct.

Typically Sapog can establish an adequate field synchronization in a few commutation steps.
If the rotor was rotating in the same direction but at a higher speed,
it will be rapidly decelerated until the speed defined by the mechanical load and the source voltage is
reached.
If it is expected that the motor will be frequently started from a non-stationary state,
it is recommended to either shorten the initial voltage ramp, or increase the initial
spin-up voltage, in order to reduce the stresses caused by the initial braking.

\subsubsection{Configuration parameters}

This section summarizes the information about the configuration parameters pertaining to the spin-up mode.
Read the preceding sections for a more detailed review.

\begin{itemize}
	\item \CfgRef{mot+spup+blnk+pm} - the duration of the extended blanking time during spin-up in permill
	(i.e. one-thousands; $10\text{\space{}permill} = 1\% $) of the current commutation period.
	\item \CfgRef{mot+spup+to+ms} - the overall spin-up timeout, in milliseconds.
	If a spin-up attempt could not be completed in this amount of time, the rotor will be considered stalled.
	\item \CfgRef{mot+spup+st+cp} - the maximum commutation period for the spin-up mode, in microseconds.
	This is also the initial commutation period.
	\item \CfgRef{mot+spup+vramp+t} - the duration of the initial voltage ramp, in seconds.
	\item \CfgRef{mot+v+spinup} - the initial source voltage $E_s$ in the beginning of the voltage ramp, in volts.
\end{itemize}

\subsection{Open loop operation}\label{sec:open_loop}

While operating in the open loop mode, Sapog accepts setpoints in the range from 0\% to 100\%,
and treats that as the fraction of the inverter supply voltage $V_\text{inv}$ to
deliver to the motor $E_s$:
\begin{equation}
	E_s = V_\text{inv}\cdot{}\mathrm{ramp}\left(\mathrm{limit}\left(S_\text{DC}\right)\right)
	\qquad\text{for open loop mode}, S_\text{DC} \in \left(0, 1\right]
\end{equation}
where $S_\text{DC}$ is the setpoint (DC stands for \emph{duty cycle}),
$\mathrm{ramp}$ is the setpoint ramp function,
and $\mathrm{limit}$ is the speed limiting P-controller, reviewed below.

The ramp function replaces abrupt changes of the setpoint with gradual changes.
Abrupt changes that are small enough to not be a hazard to the stability of the drive are
accepted by the controller directly, bypassing the ramp.
The following configuration parameters govern operation of the open loop setpoint ramp:

\begin{itemize}
	\item \CfgRef{mot+dc+slope} - the slope of the setpoint ramp, in full ranges per second.
	For example, the value of 1 allows the controller to sweep the setpoint from 0\% to 100\% in 1 second.
	The value of 10 allows the controller to sweep the setpoint from 0\% to 100\% in 0.1 seconds.
	\item \CfgRef{mot+dc+accel} - the maximum change of setpoint that will be accepted directly,
	bypassing the ramp.
	For example, the value of 0.1 allows the controller to apply step changes up to 10\% large directly.
\end{itemize}

Increase the parameters (either one or both) to improve the response characteristics of the drive.
Reduce them if rapid setpoint changes cause stability issues.

Note that the minimal acceptable setpoint is constrained by the minimum duty cycle deduced from
the parameter \CfgRef{mot+v+min}.
If the setpoint falls into $\left(0, S_{\text{DC}_\text{min}}\right]$,
where $S_{\text{DC}_\text{min}}$ is the minimum
duty cycle deduced from the parameter \CfgRef{mot+v+min},
it will be silently overridden to $S_{\text{DC}_\text{min}}$.

The maximum acceptable setpoint will be automatically limited by means of a simple
P-controller\footnote{Proportional controller, i.e. only P term of a PID controller is implemented.}
if the motor approaches the maximum operating speed permitted by the
current PWM carrier frequency, as shown on figure~\ref{pwm_maximum_speed}.

A zero setpoint is treated as the command to stop the motor.
While the motor is stopped, the VSI is disabled, allowing the shaft to rotate freely.

\subsection{RPM control loop}\label{sec:rpm_loop}

When operating in the RPM control mode, also known as RPM governor mode,
Sapog strives to maintain a specified mechanical speed.
The speed control is implemented by means of a parallel PID controller
that takes the current mechanical RPM as input and
produces the duty cycle setpoint at the output.

The nominal update rate of the PID controller $T_{\text{RPMPID}}$ is defined as follows:
\begin{equation}\label{eq:rpm_pid_control_loop_period}
	T_{\text{RPMPID}} \approx \max \left(10^{-3},T_{\text{comm}}\right)
\end{equation}
where $T_\text{comm}$ is the current commutation period.
Update of the RPM control loop is not a hard real time process, and the software is allowed to
deviate.

The RPM control loop relies on the duty cycle slope control function defined in the section
\ref{sec:open_loop}, so the related configuration parameters are still applicable.
The speed limiting P-controller that is used in the open loop mode, however,
is entirely bypassed in the RPM mode.
The maximum supported speed is dependent on the PWM carrier frequency;
see figure~\ref{pwm_maximum_speed} for details.
Equation~\ref{eq:speed_electrical_mechanical} defines the relation between electrical
and mechanical angular rates.

If the commanded setpoint is lower than the value specified by the parameter \mbox{\CfgRef{mot+rpm+min},}
the actual setpoint will be overridden to match this value.
A zero setpoint is treated as a command to stop the motor.
While the motor is stopped, the VSI is disabled, allowing the shaft to rotate freely.

The RPM PID controller does not wind up the integral term if the output is constrained by external factors,
such as:
\begin{itemize}
	\item Duty cycle ramp.
	\item Current limiting controller.
	\item Maximum RPM limit.
\end{itemize}

The configuration parameters listed below govern the operation of the RPM control loop.
The symbol $\text{DC}$ refers to the duty cycle unit ranging in $\left[-1, 1\right]$,
where negative values correspond to negative inverter source voltages $E_s$.

\begin{itemize}
	\item \CfgRef{mot+num+poles} - the number of magnetic poles on the rotor, always even, always $\geq 2$.
	This parameter is needed for the controller to convert the electrical speed $\text{RPM}_e$
	into mechanical $\text{RPM}_m$.
	\item \CfgRef{rpmctl+p} - proportional term of the RPM PID controller,
	in $\frac{\text{DC}}{\text{RPM}_m}$.
	\item \CfgRef{rpmctl+i} - integral term of the RPM PID controller,
	in $\frac{\text{DC}}{\text{second}\cdot{}\text{RPM}_m}$.
	\item \CfgRef{rpmctl+d} - derivative term of the RPM PID controller,
	in $\frac{\text{second}\cdot{}\text{DC}}{\text{RPM}_m}$.
	\item \CfgRef{mot+rpm+min} - the minimal mechanical RPM at which stable operation is guaranteed.
\end{itemize}

The overall RPM control loop output for a time step $n$ can be approximated with the following equations:
\begin{equation}
	e(n) = \omega_{m_{n}} - S_{\omega_{m_{n}}}
\end{equation}
\begin{equation}
	E_{s_n} = V_{\text{inv}_n}\cdot\mathrm{ramp}
	\left[
	\underbrace{\frac{e\left(n\right) - e\left(n-1\right)}{T_{\text{RPMPID}}} \Pi_{\text{rpmctld}}}_\text{derivative} +
	\mathrm{clamp}\left(
	\underbrace{\sum_{i=0}^n e\left(i\right) T_{\text{RPMPID}} \Pi_{\text{rpmctli}}}_\text{integral}
	\right) +
	\underbrace{e\left(n\right) \Pi_{\text{rpmctlp}}}_\text{proportional}
	\right]
\end{equation}
where:
\begin{itemize}
	\item $\mathrm{ramp}$ - the slope control function defined in section~\ref{sec:open_loop}.
	\item $\mathrm{clamp}$ - the integrator wind-up control function described in this section.
	\item $\omega_{m}$ - current speed, in mechanical RPM.
	\item $S_{\omega_{m}}$ - target speed setpoint, in mechanical RPM.
	\item $T_{\text{RPMPID}}$ - instant control loop update period.
	See equation~\ref{eq:rpm_pid_control_loop_period}.
	\item $\Pi_{\text{rpmctlp}}$ - proportional term of the PID controller.
	\item $\Pi_{\text{rpmctli}}$ - integral term of the PID controller.
	\item $\Pi_{\text{rpmctld}}$ - derivative term of the PID controller.
\end{itemize}

\subsection{Command timeout}\label{sec:setpoint_ttl}

Every setpoint supplied to the motor control logic, regardless of the operating mode, is
accompanied with a time interval that specifies its lifetime -- setpoint TTL\footnote{Short for time-to-live.}.
If the time interval expires before the next setpoint is received, Sapog will automatically
stop the motor.

This safety feature ensures that the motor will automatically stop in the event of a communication failure.

The exact duration of the setpoint TTL depends on the communication interface which provides the command.
This is reviewed in detail in the chapter~\ref{sec:communication_interfaces}.

\subsection{Current and voltage measurements}

\subsubsection{Measurement}

Sapog measures the DC inverter current $I_\text{inv}$ by means of a current shunt installed on the ground path,
the output of which is connected to the ADC via a current amplifier with a fixed gain coefficient of 10.

The inverter supply voltage $V_\text{inv}$ is measured through a resistor divider with the same
division ratio as that of the BEMF feedback circuits.
This division ratio is hard-coded per every supported hardware design and cannot be changed by the user.

Measurements of the inverter DC current $I_\text{inv}$ and voltage $V_\text{inv}$
are filtered with a first-order low-pass filter.
The corner frequency of the filter is specified by the parameter \CfgRef{mot+lpf+freq}, in hertz.

\subsubsection{Current limiting}\label{sec:current_limiting}

Sapog can limit the maximum DC bus current with the help of a simple P-controller
that activates when the instant current exceeds the limit.
The gain of the current limiting P-controller is specified via the configuration parameter
\CfgRef{mot+i+max+p}, in $\frac{\text{DC}}{\text{ampere}}$, where $\text{DC}$ is the duty cycle unit
ranging in $\left[-1, 1\right]$.
The maximum current is specified via the configuration parameter \CfgRef{mot+i+max}, in amperes.

\section{Self diagnostics}\label{sec:self_diagnostics}

Sapog contains a set of extensive diagnostic routines that can detect problems with the hardware.
The diagnostic routines are executed automatically at start up.
Additionally they can be run by an external request at any time, as long as the motor is not running.

While the tests are being executed, their progress and status information are printed to the CLI.
The user can understand the nature of detected failures by reading the printed test reports.

If at least one test fails at start up, the controller will enter the panic mode,
where it will remain until power cycled.
Learn more about this in the chapter~\ref{sec:indication}.

\subsection{Power stage test}

During the power stage test, Sapog modulates a voltage sequence shown on the table below
and then validates the acquired BEMF samples.
The following legend applies:
\begin{itemize}
	\item L - low voltage level. The low side transistor is constantly turned on.
	\item H - high voltage level. A high duty cycle PWM is modulated on the phase.
	\item F - the phase is floating. Both high and low transistors are turned off.
\end{itemize}

\begin{ZubaxCompactTable}{|c|c c c|}
	Step    & Phase A & Phase B & Phase C \\
	1       & L       & F       & F       \\
	2       & H       & F       & F       \\
	3       & F       & L       & F       \\
	4       & F       & H       & F       \\
	5       & F       & F       & L       \\
	6       & F       & F       & H       \\
\end{ZubaxCompactTable}

The voltage at the output of the BEMF feedback circuits is sampled once for each non-floating phase;
which produces six samples $Q_1 \ldots Q_6$.
Then the following assertions are verified:
\begin{equation}
	\begin{aligned}
		&Q_1 < L \\
		&Q_3 < L \\
		&Q_5 < L \\
		&Q_2 \approx Q_4 \approx Q_6
	\end{aligned}
\end{equation}
where $L$ is a specific lower threshold defined by the hardware implementation,
typically at the level corresponding to 5\% of the dynamic range of BEMF.

The test is considered passed only if all assertions are true.
It is easy to see that this test is invariant to the fact whether the motor is connected or not.

Malfunctions in the following components may cause this test to fail:
\begin{itemize}
	\item VSI transistors.
	\item VSI transistor drivers.
	\item PWM module of the microcontroller.
	\item BEMF feedback circuits.
	\item ADC module of the microcontroller.
\end{itemize}

\subsection{Phase cross-conductivity test}

The cross-conductivity test allows the controller to detect short-circuits between any of the three
phases or their BEMF feedback circuits.
During this test, the sequence shown in the table below is modulated at the output of the VSI.
The following legend applies:
\begin{itemize}
	\item H - high voltage level. A high duty cycle PWM is modulated on the phase.
	\item F - the phase is floating. Both high and low transistors are turned off.
\end{itemize}

\begin{ZubaxCompactTable}{|c|c c c|}
	Step    & Phase A & Phase B & Phase C \\
	1       & H       & F       & F       \\
	2       & F       & H       & F       \\
	3       & F       & F       & H       \\
\end{ZubaxCompactTable}

At each step, voltages of all three phases are sampled, which yields the set of samples shown in the next table.
It is assumed that the phases that are floating will be pulled down by the resistor dividers in their BEMF feedback
circuits.

\begin{ZubaxCompactTable}{|c|c c c|}
	Step    & Phase A & Phase B & Phase C \\
	1       &$Q_{1A}$ &$Q_{1B}$ &$Q_{1C}$  \\
	2       &$Q_{2A}$ &$Q_{2B}$ &$Q_{2C}$  \\
	3       &$Q_{3A}$ &$Q_{3B}$ &$Q_{3C}$  \\
\end{ZubaxCompactTable}

The following assertions are then validated:
\begin{equation}
	\begin{aligned}
		&Q_{1A} > H \qquad{} Q_{1B} < L \qquad{} Q_{1C} < L \\
		&Q_{2A} < L \qquad{} Q_{2B} > H \qquad{} Q_{2C} < L \\
		&Q_{3A} < L \qquad{} Q_{3B} < L \qquad{} Q_{3C} > H \\
	\end{aligned}
\end{equation}
where $H$ and $L$ are arbitrary hardware-defined thresholds, high and low, respectively.

Failure of at least one assertion against $H$ indicates that the corresponding phase
or its BEMF feedback circuits are not functioning correctly.
Failure of at least one assertion against $L$ indicates that there is a short circuit between
the current high phase and the phase of the column of the failed assertion.

It is easy to see that the test will always be failing as long as the motor is connected.
Sapog can recognize that the motor is connected by observing all phases shorted,
in which case the results of this test will be discarded and the test will be considered passed.

\subsection{Inverter voltage and current feedback test}

Over the course of this test, Sapog merely verifies that the measurements reported by the
current and voltage measurement blocks are sane and fall within the expected ranges,
which are hardware-defined.

This test can detect issues with the respective measurement circuits,
ADC, and the power supply of the microcontroller.

\subsection{Temperature monitoring}

Sapog periodically queries the digital temperature sensor installed on the power stage.
This information is reported outside via communication interfaces, but not used
by Sapog itself.

\chapter{Indication}\label{sec:indication}

Sapog is equipped with means of visual and audial indication that are reviewed in this chapter.

\section{RGB LED}

The RGB LED is used to indicate the status of the firmware or the status of the bootloader while it is running.
It also can be controlled externally via one of the available communication interfaces.
In the latter case, the status indication output is overridden in favor of the externally
received indication commands.

\newcommand{\ShowSolidColor}[1]{%
	{\color{#1}\rule{0.4em}{0.8em}\rule{0.4em}{0.8em}\rule{0.4em}{0.8em}\rule{0.4em}{0.8em}\rule{0.4em}{0.8em}}%
}
\newcommand{\ShowBlinkingColor}[1]{{%
		\color{#1}\rule{0.4em}{0.8em}%
		\color{black}\rule{0.4em}{0.8em}%
		\color{#1}\rule{0.4em}{0.8em}%
		\color{black}\rule{0.4em}{0.8em}%
		\color{#1}\rule{0.4em}{0.8em}%
}}

\subsection{Firmware status indication}

While the firmware is running, the following LED colors can be used to reflect its status.

\begin{ZubaxSimpleTable}{Firmware status indication via RGB LED}{|l l|X|}
	State            &                         & Meaning \\
	Solid white      & \ShowSolidColor{lightgray}& Initialization is in progress, not ready to work yet. \\
	Solid green      & \ShowSolidColor{green}  & Functioning normally. \\
	Blinking cyan    & \ShowBlinkingColor{cyan}& UAVCAN auto-enumeration is in progress, awaiting user input. \\
	Solid yellow     & \ShowSolidColor{yellow} & Controller is locked (see section~\ref{sec:stall_detection}),
	or the temperature sensor is malfunctioning. \\
	Solid red        & \ShowSolidColor{red}    & Fatal error, or self test failure.
	Use CLI to obtain detailed info. \\
\end{ZubaxSimpleTable}

\subsection{Bootloader status indication}

The bootloader starts immediately after the device is powered on.
It runs for several seconds, and then, unless there is no valid firmware to boot
or an update of the firmware is requested, the bootloader starts the firmware.
More information on the bootloader and its state indication features
is available in section~\ref{sec:bootloader}.

\subsection{External control}\label{sec:visual_indication_external_control}

The RGB LED can be controlled by external commands via the available communication interfaces.
External control overrides the state indication capabilities of the LED.

\section{Sound}\label{sec:audial_indication}

Sapog can modulate arbitrary sounds by applying specific low-power voltage patterns to the motor windings.
The resulting currents cause the windings to vibrate, which produces audible sounds at the frequency of the
applied voltage.

This feature can be invoked by an external command via the supported communication interfaces.

At start up, Sapog generates two short beeps with the following parameters:
frequency 1 kHz, duration 100 ms each, spacing 200 ms.

\begin{ZubaxSimpleTable}{Sound modulation capabilities}{|X|c c|l|}
	Parameter                            & Min & Max  & Unit \\
	Audio frequency                      & 100 & 5000 & Hz \\
	Duration of modulated sound          & 1   & 100  & ms \\
\end{ZubaxSimpleTable}

If the requested parameters fall outside of the limits specified in the table,
Sapog will automatically constrain the parameters in order to produce the closest achievable results.

Sound modulation requests are only processed if the controller is idling and the motor is stopped.
Otherwise, sound modulation requests will be silently ignored.

\chapter{Communication interfaces}\label{sec:communication_interfaces}

Sapog is equipped with multiple communication interfaces.
Not all communication interfaces can be disabled,
and therefore, more than one interface can be active at the same time.

\section{UAVCAN}
\subsection{Configuration for use}
For use with UAVCAN v1, a person needs to configure the ESC into the final system in which it will be used. This will require communicating with the UAVCAN v1 node via CAN. Through the CAN communication, some registers need to be set.

\begin{ZubaxSimpleTable}{Quick definitions / abbreviations}{|l l | X|}\label{definitions_uavcan}
	Nr & We say & We mean \\
	1 & Commander & The node which is going to be sending the ESC commands and listening to its status on different messages. \\
	2 & Subject id & This is a number allocated to a topic. This and all the rest is further explained in the UAVCAN specification. Specifically this is explained in section "Subjects and services", 2.1.1.2 on page 5/143 as of 11.03.2022.\newline\newline \url{https://uavcan.org/specification/UAVCAN_Specification_v1.0-beta.pdf}\\
	3 & Configurator / Integrator & The person who takes the steps needed to make the UAVCAN v1 ESC ready for use with the specific UAVCAN network that this ESC is being connected into. The need for this procedure is explained in the UAVCAN specification but can be summarized as trying to avoid conflicts of subject id's. Different manufacturers who are independently producing UAVCAN v1 compatible devices don't need to fix subject-ids for different messages, thus avoiding conflicts.\\
	4 & Dynamic subject id. & A subject-id that does not have a fixed port number. The port number that will be used for communication needs to be the same on both communicating devices and this is achieved through a configuration procedure carried out by the configurator. Then the devices know which port to publish or receive on.
	
\end{ZubaxSimpleTable}

\begin{ZubaxSimpleTable}{Some registers have matching counterparts}{|l | X|}\label{register_counterparts}
	ESCs register         & Explanation / Commander's Register \\
	uavcan.node.id        & This needs to be set when the device has the cyan color showing on its led, before any communication to the node can happen, the node needs a node-id. \\
	uavcan.sub.setpoint.id & uavcan.\textbf{pub}.setpoint.id \\
	uavcan.sub.readiness.id &  uavcan.\textbf{pub}.readiness.id \\
	uavcan.pub.feedback.id & uavcan.\textbf{sub}.feedback.id\\
	uavcan.pub.status.id & uavcan.\textbf{sub}.status.id\\
	uavcan.pub.dynamics.id & uavcan.\textbf{sub}.dynamics.id\\
	control\_mode\_rpm &  This register will have the default value of TRUE, which means that RPM control mode will be used and all messages sent to the dynamic port id defined in register uavcan.sub.setpoint.id will have the numbers of its array correspond to RPM values.\newline\newline
	When the value is set to FALSE then a ratiometric control-mode is used and values in the array will be interpreted as a decimal range of min to max power.\\
	uavcan.pub.power.id & uavcan.\textbf{sub}.power.id \\
	ttl\_milliseconds & Only available on the ESC, can be set by the commander.\\
	id\_in\_esc\_group & Only available on the ESC, can be set by the commander. 
	
	The commander can use this index and the uavcan.sub.setpoint.id to specify which ESC it is trying to control.
\end{ZubaxSimpleTable}

\subsection{Application (not bootloader)}
The following three tables maintain the numbers and names of registers in the same order.


\begin{ZubaxSimpleTable}{Register name and its type}{|l l | X|}\label{register_name_type}
	Nr & Name            & Register type\\
	1 & uavcan.node.id        & uavcan.node.ID \\
	2 & uavcan.node.description & uavcan.primitive.String\_1\_0 \\
	3 & uavcan.sub.setpoint.id & uavcan.primitive.array.Natural16\_1\_0 \\
	4 & uavcan.sub.readiness.id & uavcan.primitive.array.Natural16\_1\_0 \\
	5 & uavcan.pub.feedback.id & uavcan.primitive.array.Natural16\_1\_0 \\
	6 & uavcan.pub.status.id & uavcan.primitive.array.Natural16\_1\_0\\
	7 & uavcan.pub.dynamics.id & uavcan.primitive.array.Natural16\_1\_0 \\
	8 & control\_mode\_rpm & uavcan.primitive.array.Bit\_1\_0 \\
	9 & uavcan.pub.power.id & uavcan.primitive.array.Natural16\_1\_0 \\
	10 & ttl\_milliseconds & uavcan.primitive.array.Natural16\_1\_0\\
	11 & id\_in\_esc\_group &  uavcan.primitive.array.Natural16\_1\_0
\end{ZubaxSimpleTable}

\begin{ZubaxSimpleTable}{Register name mapped to its subject's type}{|l l | X|}\label{register_name_mapped_to_its_subject's_type}
	Nr & Name            & Subject's DSDL type\\
	1 & uavcan.node.id        & not a subject id \\
	2 & uavcan.node.description & not a subject id \\
	3 & uavcan.sub.setpoint.id & reg.udral.service.actuator.common.sp.Vector2\_0 \\
	4 & uavcan.sub.readiness.id & reg.udral.service.common.Readiness\_0\_1 \\
	5 & uavcan.pub.feedback.id & reg.udral.service.actuator.common.Feedback\_0\_1 \\
	6 & uavcan.pub.status.id & reg.udral.service.actuator.common.Status\_0\_1\\
	7 & uavcan.pub.dynamics.id & reg.udral.physics.dynamics.rotation.PlanarTs\_0\_1 \\
	8 & control\_mode\_rpm & not a subject id \\
	9 & uavcan.pub.power.id & reg.udral.physics.electricity.PowerTs\_0\_1 \\
	10 & ttl\_milliseconds & not a subject id\\
	11 & id\_in\_esc\_group & not a subject id
\end{ZubaxSimpleTable}

For more accurate DSDL descriptions please refer to the links in the following datatype links table.

\begin{ZubaxSimpleTable}{Register name, use case}{|l l | X|}\label{register_name_and_usecase}
	Nr & Name            & Use\\
	1 & uavcan.node.id        & Defines a node-ID. The maximum valid value is dependent on the underlying transport layer. Values lower than 128 are always valid for all transports. Refer to the specification for more info. \\
	2 & uavcan.node.description & Describing the node. A UTF8-encoded string of text. Since the string is represented as a dynamic array of bytes, it is not null-terminated. Like Pascal string.\\
	3 & uavcan.sub.setpoint.id & The dynamic subject id that Sapog will listen on to set its ESC setpoint \\
	4 & uavcan.sub.readiness.id & The dynamic subject id that Sapog will listen on to set/keep its ESC readiness\\
	5 & uavcan.pub.feedback.id & This high-rate feedback is published once immediately after a setpoint is applied. \\
	6 & uavcan.pub.status.id & Auxiliary actuator status information published at a low rate asynchronously, usually at 1 Hz. It is mostly intended for diagnostics and logging purposes.\\
	7 & uavcan.pub.dynamics.id & Torque is timestamped. \\
	8 & control\_mode\_rpm & Whether to use RPM mode (if true) or to use ratio-metric control 0 to 1 decimal power. \\
	9 & uavcan.pub.power.id & DC or AC line timestamped electric power quantities. \\
	10 & ttl\_milliseconds & Allowed time in milliseconds for the ESC to run after receiving a setpoint. \\
	11 & id\_in\_esc\_group & The id of this ESC in the group of ESCs that it belongs to. The group of the ESC is determined by the subject id of setpoints.
\end{ZubaxSimpleTable}

If any of the following datatype links become unavailable due to a changed folder structure or name of the branch used, please use this link to access the repository and navigate to the correct folder or use search, \url{https://github.com/UAVCAN/public_regulated_data_types/}

\begin{ZubaxSimpleTable}{Datatype links}{|l l | X|}\label{datatype_links}
	Nr & Name            & Link\\
	1 & uavcan.node.id        & \url{https://github.com/UAVCAN/public_regulated_data_types/blob/master/uavcan/node/ID.1.0.uavcan} \\
	2 & uavcan.node.description & \url{https://github.com/UAVCAN/public_regulated_data_types/blob/master/uavcan/primitive/String.1.0.uavcan} \\
	3 & uavcan.sub.setpoint.id & \url{https://github.com/UAVCAN/public_regulated_data_types/tree/master/reg/udral/service/actuator/common/sp} \\
	4 & uavcan.sub.readiness.id & \url{https://github.com/UAVCAN/public_regulated_data_types/blob/master/reg/udral/service/common/Readiness.0.1.uavcan}\\
	5 & uavcan.pub.feedback.id & \url{https://github.com/UAVCAN/public_regulated_data_types/blob/master/reg/udral/service/actuator/common/Feedback.0.1.uavcan}\\
	6 & uavcan.pub.status.id & \url{https://github.com/UAVCAN/public_regulated_data_types/blob/master/reg/udral/service/actuator/common/Status.0.1.uavcan}\\
	7 & uavcan.pub.dynamics.id & \url{https://github.com/UAVCAN/public_regulated_data_types/blob/master/reg/udral/physics/dynamics/rotation/PlanarTs.0.1.uavcan} \\
	8 & control\_mode\_rpm & \url{https://github.com/UAVCAN/public_regulated_data_types/blob/master/uavcan/primitive/array/Bit.1.0.uavcan} \\
	9 & uavcan.pub.power.id &  \\
	10 & ttl\_milliseconds & \url{https://github.com/UAVCAN/public_regulated_data_types/blob/master/uavcan/primitive/scalar/Natural16.1.0.uavcan} \\
	11 & id\_in\_esc\_group & \url{https://github.com/UAVCAN/public_regulated_data_types/blob/master/uavcan/primitive/scalar/Natural16.1.0.uavcan}
\end{ZubaxSimpleTable}

\begin{ZubaxSimpleTable}{Application status indication via RGB LED}{|l l|X|}
	State            &                         & Meaning \\
	Solid red        & \ShowSolidColor{red}    & Error.\\
	Solid dim green      & \ShowSolidColor{green}  & Running normally after node ID allocation
	This state is very transient. \\
	Solid cyan    & \ShowSolidColor{cyan} & UAVCAN dynamic node ID allocation started. \\
	Off              & \ShowSolidColor{black}  & The application is being started. 
\end{ZubaxSimpleTable}

\begin{ZubaxSimpleTable}{The different commands that are accepted on the ExecuteCommand.1.1 subject}{|l X|}
	Command & Explanation \\
	uavcan\_node\_ExecuteCommand\_Request\_1\_1\_COMMAND\_RESTART & Restarts the board\\
	uavcan\_node\_ExecuteCommand\_Request\_1\_1\_COMMAND\_STORE\_PERSISTENT\_STATES & Stores configuration\\
	uavcan\_node\_ExecuteCommand\_Request\_1\_1\_COMMAND\_BEGIN\_SOFTWARE\_UPDATE & Starts a software update.\\
\end{ZubaxSimpleTable}
These commands are described in \url{https://github.com/UAVCAN/public_regulated_data_types/blob/master/uavcan/node/435.ExecuteCommand.1.1.uavcan}

\subsection{Bootloader (not application)}
\section{CLI}

\subsection{Overview}

Sapog exposes a plain text command line interface (CLI) on its serial port.
The CLI can be used for user interaction directly; also it can be used as a machine-to-machine command interface,
as the command syntax is quite simple.

The serial port interface operates at 115200 baud, 8 bit per word, no parity control, one stop bit
(115200-8N1).

The CLI line termination sequence is \verb|\r\n| (CR, LF).

Barring any special preferences,
it is recommended to use the software listed in table~\ref{cli_software_recommendations}
for accessing the CLI from the user's PC.

\begin{ZubaxSimpleTable}{CLI terminal emulators}{|l l X|}\label{cli_software_recommendations}
	PC OS            & Application         & Note \\
	GNU/Linux        & \texttt{picocom}    & A simple and robust command-line tool.
	It should be preferred over the popular
	\texttt{minicom} and \texttt{screen}. \\
	MS Windows       & PuTTY               & A simple multi-purpose GUI terminal emulator.
	The HyperTerminal tool supplied by default with some versions
	of Windows should be avoided. \\
\end{ZubaxSimpleTable}

\subsection{Boot logging}

While in the process of booting, Sapog prints human-readable diagnostic messages via the serial port.
The printed messages, among other things, include the reports of the start-up self-test procedures
described in section~\ref{sec:self_diagnostics}.

Normally the boot-up messages should only be of interest for troubleshooting purposes.

\subsection{Panic reporting}

If Sapog encounters a catastrophic failure that renders it dysfunctional
(e.g. a memory access violation, stack overflow, failure of the computing hardware, etc),
it uses the serial interface as a last resort status reporting option.
In this case, Sapog will dump as much status information as possible into the serial port,
after which it will lock up and wait for the watchdog timer to reset it.

\subsection{CLI commands}

This section documents the CLI commands that can be of interest to the end user.
Some commands that are not intended for use in production are intentionally omitted from this reference.

\subsubsection{help}

Prints the list of all available commands.

\subsubsection{cfg}

This command provides access to the configuration parameter storage.
The configuration parameters and their non-volatile storage are described in more detail in the chapter
\ref{sec:configuration_parameters}.

Syntax:
\begin{itemize}
	\item \verb|cfg list| - list all configuration parameters, their current values,
	acceptable value intervals, and default values.
	
	\item \verb|cfg save| - save the current configuration into the non-volatile storage.
	This command is redundant and normally need not be used, because Sapog commits all configuration
	into the non-volatile storage automatically upon modification.
	
	\item \verb|cfg erase| - erase the current configuration and reset the non-volatile memory to
	the factory defaults.
	
	\item \verb|cfg set <name> <value>| - assign the configuration parameter named \verb|<name>| the value
	\verb|<value>|. For example: \verb|cfg set foo 42|.
	The non-volatile storage will be updated automatically.
\end{itemize}

The syntax \verb|cfg list| prints one parameter per line, where the line is formatted as follows:

\verb|name = value [min, max] (default)|

The number of whitespaces between tokens may vary.

Floating point parameters are always reported with the point symbol (\verb|.|),
which allows one to distinguish them from integer and boolean typed parameters.

Boolean parameters are reported as integers, where 1 represents the logical true and
0 represents the logical false.
They can be distinguished from integer parameters by their minimum and maximum values being 0 and 1,
respectively.

Whenever a parameter is assigned a new value, the device verifies if the new value is within the
acceptable limits.
If it is, the new value is assigned. Otherwise, the old value is retained.
Afterwards the device prints out the resulting value of the parameter in the following format:

\verb|name = value|

The number of whitespaces between tokens may vary.

The response can be used to check whether the new value was accepted by the device or not.

\subsubsection{reboot}

Reboot the firmware.
If the motor is running, it will be stopped first.

\subsubsection{beep}

Modulate sound as described in section~\ref{sec:audial_indication}.

Syntax:
\begin{itemize}
	\item \verb|beep| - emit sound of the default duration at the default frequency.
	\item \verb|beep <freq>| - modulate sound of the default duration at the frequency \verb|<freq>|
	specified in hertz. For example: \verb|beep 1000|.
	\item \verb|beep <freq> <dur>| - modulate sound of the specified duration \verb|<dur>| in milliseconds
	at the frequency \verb|<freq>| specified in hertz. For example: \verb|beep 5000 200|.
\end{itemize}

\subsubsection{stat}

Print the current state of the motor controller in a human-readable form.
At least the following information will be displayed:
\begin{itemize}
	\item Inverter supply voltage.
	\item Inverter DC current.
	\item Mechanical RPM.
	\item Actual duty cycle (not the setpoint).
	\item The number of failures since the motor has started.
\end{itemize}

\subsubsection{test}

Execute the self-test routines, described in section~\ref{sec:self_diagnostics}.
The output will be printed in a human-readable format.

\subsubsection{dc}

Set the open loop control setpoint.
See section~\ref{sec:open_loop} for more information about the open loop control mode.

The setpoint TTL when using this command is fixed at 30 seconds.
More information on the setpoint TTL feature is provided in section~\ref{sec:setpoint_ttl}.

By default this command is locked for safety reasons.
In order to unlock it, execute the command
``\verb|dc arm|'' once, and it will stay unlocked until the firmware is rebooted.

Syntax:
\begin{itemize}
	\item \verb|dc| - assign a zero setpoint, which will stop the motor if it is running.
	\item \verb|dc arm| - unlock the command.
	\item \verb|dc <duty cycle>| - set the specified duty cycle setpoint.
	The setpoint is a real number in the interval $\left[0, 1\right]$.
	For example: \verb|dc 0.5| assigns the 50\% duty cycle.
\end{itemize}

\subsubsection{rpm}

Set the RPM control loop setpoint.
See section~\ref{sec:rpm_loop} for more information about the RPM control mode.

The setpoint TTL when using this command is fixed at 30 seconds.
More information on the setpoint TTL feature is provided in section~\ref{sec:setpoint_ttl}.

By default this command is locked for safety reasons.
In order to unlock it, execute the command
``\verb|rpm arm|'' once, and it will stay unlocked until the firmware is rebooted.

Syntax:
\begin{itemize}
	\item \verb|rpm| - assign a zero setpoint, which will stop the motor if it is running.
	\item \verb|rpm arm| - unlock the command.
	\item \verb|rpm <RPM>| - set the specified RPM setpoint.
	The setpoint is an integer number in the interval $\left[0, 65535\right]$.
	For example: \verb|rpm 5000| assigns the target setpoint of 5000 mechanical RPM.
\end{itemize}

\subsubsection{md}

Print the internal motor controller status information in a human-readable format.
The output of this command is also printed automatically when the rotor stalls.

\subsubsection{zubax\_id}\label{sec:cli_zubax_id}

This command reports the complete information identifying this particular product:
type, version information, make and model.
The information is printed in a machine-readable yet human-friendly
YAML\footnote{\url{https://en.wikipedia.org/wiki/YAML}} format.

An example output is shown in the following listing.
The meaning of each well-defined field is explained in table~\ref{zubax_id_fields_table}.
Fields that are not intended for production use are intentionally left undocumented.
Note that the number of fields and their ordering may vary.

\begin{minted}{yaml}
	product_id   : 'io.px4.sapog'
	product_name : 'PX4 Sapog'
	sw_version   : '2.0'
	sw_vcs_commit: 7672820
	sw_build_date: Apr 29 2017
	hw_version   : '1.1'
	hw_unique_id : 'M//UBUdHNTZUVxlDAAAAAA=='
	hw_signature : 'NX3Y05Y1a/hDNajPyMF5dojuAB9r1HmUxZ3UVU8g39C9L5pjAZMVKdZ7yI1CetvumOJZHIjRYkHOBmMAEKvRU/3mpG\
	+KgUND6yGpoHQ5aa+OK+LIcOw2LC+BaUmAPS6Hy4bxOvuARb062QruQlrOXyVc/vs28JtcBzOZo/b/OxY='
	hw_info_url  : 'http://device.zubax.com/device_info?uid=33FFD405474735365457194300000000'
\end{minted}

\begin{ZubaxSimpleTable}{Zubax ID fields}{|l X|}\label{zubax_id_fields_table}
	Field name              & Meaning \\
	
	\texttt{product\_id}    & Product type identifier string. It is always set to \texttt{io.px4.sapog}.
	The same string is reported via UAVCAN as the node name string. \\
	
	\texttt{sw\_version}    & Firmware version number in the form ``major.minor''. \\
	
	\texttt{sw\_vcs\_commit}& Firmware patch ID as the short git commit hash.
	The abbreviation VCS stands for ``version control system''.
	This number allows one to determine the exact revision of the firmware that is currently running. \\
	
	\texttt{sw\_build\_date}& Firmware build date in the form ``\texttt{mmm dd yyyy}''. \\
	
	\texttt{hw\_version}    & Hardware version number in the form ``major.minor''.
	Refer to the documentation provided with your hardware to learn what version number it reports to Sapog. \\
	
	\texttt{hw\_unique\_id} & The 128-bit unique ID of this specific hardware instance.
	The UID is represented either as a hexadecimal string, or as a Base64 encoded string.\\
	
	\texttt{hw\_signature}  & The certificate of authenticity (CoA) of this specific hardware instance
	encoded in a Base64 string.
	This field will not be present if the current hardware instance does not have any CoA installed.
	Refer to the documentation provided with your hardware to find out what CoA is used, if any. \\
	
	\texttt{hw\_info\_url}  & Link to the web page that contains the test report, origin information,
	traceability data, and other important information about this specific hardware instance provided by
	the vendor. This field may not be populated if the hardware does not have any CoA installed. \\
	
\end{ZubaxSimpleTable}

Vendors that use Sapog for their own products can use this command to install their own CoA on the hardware.
Beware that the CoA can be written only once.
The following form is used to write the CoA: \verb|zubax_id <CoA>|,
where \verb|<CoA>| is a placeholder for a Base64-encoded certificate of authenticity.
More information about using Sapog in custom products is provided in the chapter~\ref{sec:porting_guide}.

Syntax:
\begin{itemize}
	\item \verb|zubax_id| - print the Zubax ID YAML structure documented above.
	\item \verb|zubax_id <CoA>| - install the vendor's certificate of authenticity \verb|<CoA>| encoded in Base64.
	This command can be executed only once.
\end{itemize}

\subsubsection{uavcan}

Print the status information of the local UAVCAN stack in a human-readable format.
This command may take several seconds to execute, depending on the state of the UAVCAN stack.

\section{RCPWM}

\subsection{Overview}

RCPWM is a simple analog interface that encodes the command in a series of repeating pulses.
The commanded setpoint is proportional to the duration of the pulses,
whereas the spacing between the pulses and their interval do not carry any useful information.
The principle is illustrated on figure~\ref{rcpwm_signal_plot}.

\begin{figure}[hbt]
	\centering
	\includegraphics[width=0.7\textwidth]{rcpwm_signal_plot}
	\caption{RCPWM signal
		\label{rcpwm_signal_plot}}
\end{figure}

The RCPWM interface is widely used in robotics and in flight control systems of small UAV.
Its obvious downsides are the lack of any kind of feedback from the controlled system,
very low throughput,
high latency (in excess of 2 milliseconds in certain scenarios, and never lower than 1 millisecond),
and general physical inefficiency due to the need to use one independent electrical connection
per controlled system.

\subsection{Functional description}

When the RCPWM interface is enabled, Sapog demodulates the RCPWM signal into the open loop mode setpoints,
and feeds them into the motor controller.
Setpoints received via RCPWM are accepted immediately without pre-filtering or any other preprocessing.

In order to be understood by Sapog,
the RCPWM signal should match the requirements listed in table~\ref{rcpwm_constraints}.

\begin{ZubaxSimpleTable}{RCPWM interface constraints}{|c X|c c|c|}\label{rcpwm_constraints}
	Symbol                               & Parameter            & Min   & Max & Unit \\
	$\Delta{}t_{\text{HL}_\text{RCPWM}}$ & RCPWM pulse duration & 0.5   & 3   & millisecond \\
	$\Delta{}t_{\text{LH}_\text{RCPWM}}$ & RCPWM pulse spacing  & 0.01  & 50  & millisecond \\
\end{ZubaxSimpleTable}

If the RCPWM pulse duration $\Delta{}t_{\text{HL}_\text{RCPWM}}$ is not within the specified limits,
Sapog treats it as if there was no modulation at all.

The setpoint TTL for RCPWM commands is fixed at 100 milliseconds.
This implies that the motor will be stopped in 100 milliseconds after the RCPWM modulation is gone.
More information on the setpoint TTL feature is provided in section~\ref{sec:setpoint_ttl}.

The configuration parameter \CfgRef{pwm+enable} specifies whether the RCPWM interface is active.
If the interface is disabled, Sapog will ignore the state of the RCPWM input line.
If the interface is enabled, care should be taken to ensure that no conflicting commands
are being issued via other interfaces, otherwise the controller may exhibit an erratic behavior.

The configuration parameters \CfgRef{pwm+min+usec} and \CfgRef{pwm+max+usec} define the pulse duration
that corresponds to the minimum (zero) and the maximum (100\%) setpoints, respectively, in microseconds.
Pulses shorter than \CfgRef{pwm+min+usec} are interpreted as a zero setpoint;
likewise, pulses that are longer than \CfgRef{pwm+max+usec} are interpreted as the maximum setpoint.

Sapog applies a start-stop hysteresis that works as follows: if the motor is stopped, it will not start
until the commanded setpoint exceeds 6\%.
If the motor is running, it will not stop until the commanded setpoint falls below 3\%.

A safety feature will not allow the motor to start if the commanded setpoint exceeds 40\%.
Therefore, combining the effects of the hysteresis feature and the safety threshold,
it can be seen that the motor will start only if the commanded setpoint falls into the interval
$\left(6\%, 40\%\right)$.

The setpoint handling logic is illustrated on figure~\ref{rcpwm_setpoint_handling}.

\begin{figure}[hbt]
	\centering
	\includegraphics[width=0.8\textwidth]{rcpwm_setpoint_handling}
	\caption{RCPWM setpoint handling
		\label{rcpwm_setpoint_handling}}
\end{figure}

\chapter{Configuration parameters}\label{sec:configuration_parameters}

This chapter summarizes all configuration parameters available in Sapog.

The length of configuration parameter names does not exceed 16 characters,
which ensures compatibility with the MAVLink protocol version 1.

\section{Non-volatile configuration storage}

All configuration parameters are stored in a non-volatile memory
that retains its contents across power cycles.

Modification of any single parameter will trigger Sapog to commit all of them into the non-volatile
memory one second after the last modification has taken place.
The one second delay ensures that no excessive modifications of the non-volatile memory will be carried out
when multiple configuration parameters are changed at the same time.

Sapog will not write or erase the non-volatile memory as long as the motor is running, in order to
avoid interference with the hard real time motor control algorithms.
If configuration was changed while the motor is running,
a write operation will be postponed until the motor is stopped.

If the device is turned off while the configuration storage was being updated,
the stored configuration data may get damaged.

The stored configuration is read from the non-volatile memory once upon boot-up.
If Sapog detects that the stored configuration data is damaged,
it will automatically revert it to the factory default state.
Sapog can always reliably detect a damage of the stored configuration data,
so it is guaranteed that an invalid configuration can never be loaded.

\section{Firmware upgrade considerations}

Configuration parameters of different revisions of the firmware may be incompatible with each other.
For instance, some configuration parameters may be added, removed, or their value intervals may be changed.

Sapog always checks whether the stored configuration data is compatible with the current firmware revision.
If it is detected that the stored configuration cannot be applied to the current version of the firmware,
Sapog will automatically revert to the factory default configuration.

Keep these considerations in mind when upgrading the firmware.

\section{Configuration parameter index}

The minimum, maximum, and default values provided in the table are shown for exemplary purposes only,
and they are \emph{not expected to be valid} for all firmware revisions that this document applies to.
The value intervals and default values may change in newer firmware revisions; furthermore,
they can also depend on which particular hardware Sapog is running on.

The following abbreviations are adopted for the tables below:
\begin{description}
	\item[TEA] Short for ``takes effect at''. This field specifies when the newly specified parameter value will
	take effect.
	\item[Def.] Short for ``default value''.
\end{description}

% name, unit, takes effect at, min, max, default, note
\newcommand\CfgParamIndexEntry[7]{%
	\CfgDef{#1} & \footnotesize{#2} & \footnotesize{\CfgListReferences{#1}} & \footnotesize{#3} &
	\footnotesize{#4} & \footnotesize{#5} & \footnotesize{#6} &
	\footnotesize{#7} \tabularnewline
}%

\newenvironment{CfgParamIndex}[1]{%
	\begin{ZubaxTableWrapper}{#1}
		\setlength\tabcolsep{2.5pt}
		\begin{ZubaxWrappedTable}{@{} l l l l | c c c | X @{}}
			Name & Unit & Pages & TEA & Min & Max & Def. & Description \\
		}{%
		\end{ZubaxWrappedTable}
	\end{ZubaxTableWrapper}
}

\subsection{Motor control configuration parameters}

\begin{CfgParamIndex}{Motor control configuration parameters}
	\CfgParamIndexEntry{mot+pwm+dt+ns}{nanosecond}{boot}{400}{800}{}{PWM dead time}
	\CfgParamIndexEntry{mot+pwm+hz}{hertz}{boot}{20000}{75000}{60000}{PWM carrier frequency}
	\CfgParamIndexEntry{mot+spup+blnk+pm}{permill}{boot}{1}{300}{100}{Min blanking time during spin-up}
	\CfgParamIndexEntry{mot+spup+to+ms}{millisecond}{boot}{100}{9000}{5000}{Spin-up timeout}
	\CfgParamIndexEntry{mot+spup+st+cp}{microsecond}{boot}{10000}{300000}{100000}{Spin-up max/initial comm. period}
	\CfgParamIndexEntry{mot+comm+per+max}{microsecond}{boot}{1000}{10000}{4000}{Max comm. period for normal mode}
	\CfgParamIndexEntry{mot+zc+fails+max}{}{boot}{6}{300}{20}{Stall detection threshold}
	\CfgParamIndexEntry{mot+bemf+range}{percent}{boot}{10}{100}{90}{Valid BEMF magnitude}
	\CfgParamIndexEntry{mot+bemf+win+den}{}{boot}{3}{8}{4}{BEMF sampling window length denominator}
	\CfgParamIndexEntry{mot+blank+usec}{microsecond}{boot}{10}{300}{40}{Comm. blanking time for normal mode}
	\CfgParamIndexEntry{mot+tim+cp+min}{microsecond}{boot}{100}{50000}{600}{Advance angle interpolation point X1}
	\CfgParamIndexEntry{mot+tim+cp+max}{microsecond}{boot}{100}{50000}{300}{Advance angle interpolation point X2}
	\CfgParamIndexEntry{mot+tim+adv+max}{angular degree}{boot}{0}{29}{15}{Advance angle interpolation point Y1}
	\CfgParamIndexEntry{mot+tim+adv+min}{angular degree}{boot}{0}{20}{5}{Advance angle interpolation point Y2}
	\CfgParamIndexEntry{rpmctl+p}{$\frac{\text{dutycycle}}{\text{RPM}_m}$}{boot}{0}{1}{0.0001}%
	{RPM loop PID proportional gain}
	\CfgParamIndexEntry{rpmctl+i}{$\frac{\text{dutycycle}}{\text{second}\cdot\text{RPM}_m}$}{boot}{0}{10}{0.001}%
	{RPM loop PID integral gain}
	\CfgParamIndexEntry{rpmctl+d}{$\frac{\text{second}\cdot\text{dutycycle}}{\text{RPM}_m}$}{boot}{0}{1}{0}%
	{RPM loop PID derivative gain}
	\CfgParamIndexEntry{mot+stop+thres}{}{boot}{1}{100}{7}{Lock-up threshold}
	\CfgParamIndexEntry{mot+lpf+freq}{hertz}{boot}{1}{200}{20}{Current/voltage low-pass filter corner frequency}
	\CfgParamIndexEntry{mot+i+max}{ampere}{boot}{1}{60}{20}{Maximum inverter current}
	\CfgParamIndexEntry{mot+i+max+p}{$\frac{\text{dutycycle}}{\text{ampere}}$}{boot}{0.01}{2}{0.2}%
	{Current limiter proportional gain}
	\CfgParamIndexEntry{mot+rpm+min}{RPM}{boot}{50}{5000}{1000}{Minimum setpoint for RPM control loop}
	\CfgParamIndexEntry{ctl+dir}{}{boot}{0}{1}{0}{Direction of rotation: 0 - forward, 1 - reverse}
	\CfgParamIndexEntry{mot+num+poles}{}{boot}{2}{100}{14}{Number of magnetic poles on the rotor}
	\CfgParamIndexEntry{mot+dc+slope}{$\frac{\text{dutycycle}}{\text{second}}$}{boot}{0.1}{20}{5}%
	{Duty cycle ramp steepness}
	\CfgParamIndexEntry{mot+dc+accel}{duty cycle}{boot}{0.001}{0.5}{0.09}{Duty cycle ramp bypass threshold}
	\CfgParamIndexEntry{mot+spup+vramp+t}{second}{boot}{0}{10}{3}{Duration of spin-up voltage ramp}
	\CfgParamIndexEntry{mot+v+spinup}{volt}{boot}{0.01}{10}{0.5}{Initial voltage of spin-up voltage ramp}
	\CfgParamIndexEntry{mot+v+min}{volt}{boot}{0.5}{10}{2.5}{Min inverter source voltage for normal mode}
\end{CfgParamIndex}

\subsection{UAVCAN interface configuration parameters}

\begin{CfgParamIndex}{UAVCAN interface configuration parameters}
	\CfgParamIndexEntry{esc+index}{}{boot}{0}{15}{0}{UAVCAN index of this ESC}
	\CfgParamIndexEntry{cmd+ttl+ms}{millisecond}{boot}{100}{5000}{200}{Setpoint timeout}
	\CfgParamIndexEntry{cmd+start+dc}{duty cycle}{boot}{0.01}{1}{1}{Spin-up safety threshold}
	\CfgParamIndexEntry{uavcan+node+id}{}{boot}{0}{125}{0}{Node ID, zero for dynamic allocation}
	\CfgParamIndexEntry{light+index}{}{boot}{0}{255}{0}{UAVCAN index of the RGB LED}
\end{CfgParamIndex}

\subsection{RCPWM interface configuration parameters}

\begin{CfgParamIndex}{RCPWM interface configuration parameters}
	\CfgParamIndexEntry{pwm+max+usec}{microsecond}{boot}{1800}{2200}{2000}{Full throttle pulse duration}
	\CfgParamIndexEntry{pwm+min+usec}{microsecond}{boot}{800}{1200}{1000}{Zero throttle pulse duration}
	\CfgParamIndexEntry{pwm+enable}{}{boot}{0}{1}{0}{Enable this interface}
\end{CfgParamIndex}

\subsection{Forced rotation detection configuration parameters}

The configuration parameters listed in this group are deprecated and scheduled for removal in one
of the future revisions of the firmware.

\begin{CfgParamIndex}{Forced rotation detection configuration parameters}
	\CfgParamIndexEntry{enum+max+step}{implementation defined}{undefined}{2000}{100000}{50000}{Not for production use}
	\CfgParamIndexEntry{enum+steps}{implementation defined}{undefined}{6}{200}{20}{Not for production use}
	\CfgParamIndexEntry{enum+bemf}{implementation defined}{undefined}{5}{500}{20}{Not for production use}
\end{CfgParamIndex}

\chapter{Embedded bootloader}\label{sec:bootloader}

\section{Overview}

Sapog employs the PX4 Brickproof Bootloader - an ultra compact open source UAVCAN bootloader
designed for ROM-limited embedded applications.

The bootloader starts immediately after the device is powered on.
Having started, the bootloader checks if there is a valid application image that can be executed.
If there is one, the bootloader measures a 5 second timeout since the point of its initialization,
and once the timeout has expired, the bootloader starts the application, unless an external
entity has requested the bootloader to download a new application image.
If there is no valid application image found, the bootloader will wait forever for a request
to download a new application image.

The procedure of firmware update over UAVCAN is defined in the UAVCAN specification at the website
\url{http://uavcan.org}.

\section{Status indication}

While the bootloader is running, the following RGB LED behaviors are defined.

\begin{ZubaxSimpleTable}{Bootloader status indication via RGB LED}{|l l|X|}
	State            &                         & Meaning \\
	Solid red        & \ShowSolidColor{red}    & Error. Possible reasons: could not erase the old
	firmware image; the file server returned an invalid
	or unexpected response; file server response has timed
	out; the CRC of the firmware image is invalid. \\
	Solid white      & \ShowSolidColor{lightgray}& Just started, initialization is in progress.
	This state is very transient. \\
	Solid yellow     & \ShowSolidColor{yellow} & CAN auto bit rate detection. \\
	Solid green      & \ShowSolidColor{green}  & CAN auto bit rate detection finished.
	This state is very transient. \\
	Solid magenta    & \ShowSolidColor{magenta}& UAVCAN dynamic node ID allocation started. \\
	Solid blue       & \ShowSolidColor{blue}   & UAVCAN dynamic node ID allocation finished. \\
	Solid cyan       & \ShowSolidColor{cyan}   & Firmware image is being downloaded. \\
	Off              & \ShowSolidColor{black}  & The application is being started. \\
\end{ZubaxSimpleTable}

\section{UAVCAN interface description}

\subsection{Overview}

The bootloader operates on the CAN1 interface only.
The CAN2 interface is not used while the bootloader is running.
It is therefore mandatory to use CAN1 as the primary communication interface
rather than CAN2.

\subsection{Initialization}

Having started, the bootloader attempts to detect the bit rate of the CAN bus it is connected to
(if connected at all), unless it was started by a request sent to the application, in which case,
if the application already knew the bit rate of the bus, it will pass it to the bootloader,
allowing it to skip the bit rate detection procedure.

The bootloader can detect the following bit rates:
\begin{itemize}
	\item 1 Mbit/s
	\item 500 kbit/s
	\item 250 kbit/s
	\item 125 kbit/s
\end{itemize}

Having successfully detected the bit rate, the bootloader will attempt to obtain a UAVCAN node ID,
unless it was provided by the application.
In the latter case, the bootloader will simply re-use the node ID provided by the application.

The initialization can be interrupted at any time due to expiration of the boot timeout.
Should that happen, the bootloader will pass the bus parameters it has detected,
which are either the bit rate, the node ID, or both, to the application,
in order to speed up its initialization.

\subsection{Node status and info reporting}

The message \verb|uavcan.protocol.NodeStatus| is reported by the bootloader with the field
\verb|mode| set to \verb|MODE_INITIALIZATION| while the initialization is in progress,
or to \verb|MODE_SOFTWARE_UPDATE| while the update is in progress.
The field \verb|status| can be set either to \verb|STATUS_OK| or \verb|STATUS_CRITICAL|;
the latter is used only if the update process has failed.

The service \verb|uavcan.protocol.GetNodeInfo| is responded to with all the fields of the nested structure
\verb|software_version| zeroed, unless there is a valid application image found,
in which case the fields will be set according to the information gathered from the firmware image.
The fields of the nested structure \verb|hardware_version| are always all set correctly,
except for the field \verb|certificate_of_authenticity|, which is always empty, even if there is
a valid certificate installed.

\subsection{Firmware update}

\subsubsection{Initialization}

Having received an update request \verb|uavcan.protocol.file.BeginFirmwareUpdate|,
the bootloader verifies that the firmware image has an appropriate size by invoking
\verb|uavcan.protocol.file.GetInfo|.
If the size of the file appears to be incorrect,
or if the service call could not be completed after up to three retries,
the bootloader aborts the process with an error.
Note that if the process fails at this point, the old firmware remains intact.

\subsubsection{ROM preparation}

Next, the bootloader erases the old firmware image.
From this point on, there is no valid application that can be booted.

\subsubsection{Image downloading}

Having prepared the ROM for writing the new image, the bootloader downloads the new
firmware image by invoking the service \verb|uavcan.protocol.file.Read|
repeatedly while advancing the file read offset.
If a read request times out in the process, the bootloader will try again up to three times.
Shall all the attempts fail, the bootloader will abort the process and report failure.

\subsubsection{Finalization}

When the new image is fully downloaded, the bootloader verifies its integrity and
aborts the process if there are issues detected.
Otherwise, the application will be started immediately.
The bootloader will pass the detected CAN bus bit rate and the node ID to the application
in order to accelerate its initialization.

\subsection{Logging}

The bootloader emits the message \verb|uavcan.protocol.debug.LogMessage|
at certain moments during the update in order to indicate its progress or report issues.

The text field of the log messages always contains two ASCII letters.
The first letter indicates the action,
the second letter indicates the result of the action.
The log source name is always set to the string ``\verb|Boot|''.

\begin{ZubaxSimpleTable}{UAVCAN bootloader logging -- action codes}{|c X|}
	Code & Meaning \\
	\texttt{I} & Initialization. \\
	\texttt{G} & Invoking \texttt{uavcan.protocol.file.GetInfo}. \\
	\texttt{E} & Erasing ROM. \\
	\texttt{R} & Invoking \texttt{uavcan.protocol.file.Read}. \\
	\texttt{P} & Writing ROM. \\
	\texttt{V} & Verification of the downloaded image. \\
	\texttt{F} & Finalization. \\
\end{ZubaxSimpleTable}

\begin{ZubaxSimpleTable}{UAVCAN bootloader logging -- status codes}{|c X|}
	Code & Meaning \\
	\texttt{s} & Started \\
	\texttt{f} & Failed \\
	\texttt{o} & OK \\
\end{ZubaxSimpleTable}

\subsection{Data type summary}

\begin{ZubaxSimpleTable}{UAVCAN bootloader broadcast messages}{|l l l X|}
	Data type name                                         & Frequency, Hz & Tr. priority & Note \\
	\texttt{uavcan.protocol.NodeStatus}                    & 1          & 30 (low)     & \\
	\texttt{uavcan.protocol.dynamic\_node\_id.Allocation}  & Aperiodic  & 30 (low)     & Used only during the
	initialization. \\
	\texttt{uavcan.protocol.debug.LogMessage}              & Aperiodic  & 30 (low)     & \\
\end{ZubaxSimpleTable}

\begin{ZubaxSimpleTable}{UAVCAN bootloader subscribed messages}{|l X|}
	Data type name                                         & Note \\
	\texttt{uavcan.protocol.dynamic\_node\_id.Allocation}  & Used only during the initialization. \\
\end{ZubaxSimpleTable}

\begin{ZubaxSimpleTable}{UAVCAN bootloader provided services}{|l X|}
	Data type name                                         & Note \\
	\texttt{uavcan.protocol.GetNodeInfo}                   & Software version identification fields will be
	zeroed if there is no valid application image
	present.\\
	\texttt{uavcan.protocol.file.BeginFirmwareUpdate}      & \\
\end{ZubaxSimpleTable}

\begin{ZubaxSimpleTable}{UAVCAN bootloader invoked services}{|l l X|}
	Data type name                                         & Tr. priority & Note \\
	\texttt{uavcan.protocol.file.GetInfo}                  & 30 (low)     & Invoked once before downloading
	the firmware file. \\
	\texttt{uavcan.protocol.file.Read}                     & 30 (low)     & Used to download the firmware file. \\
\end{ZubaxSimpleTable}

\chapter{Tuning guide}

\section{Overview}

This chapter describes how to configure Sapog for optimal performance with a given motor
in a particular application.
The reader is assumed to have read the previous parts of the reference manual.

Further help can be gained via \url{https://forum.zubax.com}.

\section{Basic configuration}\label{sec:basic-configuration}

\subsection{Number of rotor magnetic poles}

The number of rotor magnetic poles is required for the closed-loop speed control and
measurement of the mechanical speed of the rotor.

The number of magnetic poles can be obtained from the motor specification;
alternatively, it can be obtained by counting the number of magnets on the rotor.

The relevant configuration parameter is \CfgRef{mot+num+poles}.

\subsection{Maximum current}

The parameter \CfgRef{mot+i+max} should be configured correctly to avoid overloading the motor.
The maximum current allowed for the motor should be provided in the motor specification.
For more information, refer to section~\ref{sec:current_limiting}.

\subsection{Low speed operation}

The default settings may be unsuitable for maintaining a stable operation of certain motors at low speeds.
This especially affects large, high-power motors.

The parameters \CfgRef{mot+comm+per+max} and \CfgRef{mot+v+min} should be adjusted in the first
order if necessary.

\subsection{Response characteristics}

The settings of the duty cycle ramp should be adjusted for the target application.
Please refer to the sections~\ref{sec:open_loop} and~\ref{sec:rpm_loop} for details.

\section{Advanced tuning}

\subsection{Field weakening}

The field weakening feature (documented in section~\ref{sec:field_weakening}) is governed by the following
set of configuration parameters:
\CfgRef{mot+tim+adv+max}, \CfgRef{mot+tim+adv+min}, \CfgRef{mot+tim+cp+min}, \CfgRef{mot+tim+cp+max}.

The angles are specified in the electrical degrees of rotor position;
one electrical degree is equivalent to 3 phase degrees.
For example, 90\textdegree{} of phase advance is equivalent to 30\textdegree{} of electrical advance.

It is recommended to follow the algorithm described below to configure the advance angle settings:

\begin{enumerate}
	\item Find out the idle commutation period of your motor.
	The commutation period (in seconds) can be derived from RPM using the following formula:
	$20 / (\text{NumPoles}\times\text{RPM})$, where \textbf{NumPoles} is the number of magnetic poles
	of the rotor, and \textbf{RPM} is the angular velocity of the rotor in revolutions per minute.
	
	\item Assign \CfgRef{mot+tim+cp+min} the idle commutation period computed in the step 1.
	
	\item Find out the commutation period when the motor is working at the nominal RPM.
	Use the formula defined in the step 1 if necessary.
	
	\item Assign \CfgRef{mot+tim+cp+max} the commutation period computed in the step 3.
	
	\item Assign \CfgRef{mot+tim+adv+max} the desired advance angle at the nominal load.
\end{enumerate}

\chapter{Porting guide}\label{sec:porting_guide}

\section{Overview}

This section contains the information necessary for implementation of custom hardware designs compatible with the
Sapog firmware.

The reader should be familiar with the firmware codebase in order to be able to implement a successful
custom hardware design compatible with Sapog.

Questions concerning custom designs can be asked via \url{https://forum.zubax.com}.

\section{Certificate of authenticity}

Sapog can store an arbitrary chunk of data up to 256 bytes large in a dedicated ROM area.
This data is supposed to be the certificate of authenticity (CoA) provided by the hardware vendor.

The CoA can be written using the command \verb|zubax_id| described in the section
\ref{sec:cli_zubax_id}.
Beware that the CoA can be written only once.

\section{Hardware design guidelines}

\subsection{Requirements}

\begin{enumerate}
	\item The state estimation methods used in Sapog require a high-speed inverter.
	When choosing the power transistors and their drivers, keep the dynamic characteristics in mind:
	the total switching time should be under 200 nanoseconds.
	
	\item The RC filters in the phase voltage feedback circuits should be designed so that their cutoff frequency
	is around 19 kHz and the output impedance does not exceed 8 $\text{k}\Omega$.
	
	\item Unless the maximum supply voltage exceeds 25 V (Li-ion $\text{LiCoO}_2$ 6S),
	the DC link voltage measurement divider should not be changed from the default specified in
	the reference schematic.
\end{enumerate}

\subsection{Reference schematic}

This section contains the schematic of the reference hardware design that can be used as a basis for custom hardware
designs compatible with Sapog.
The design files and additional information can be gathered from the GitHub repository at
\url{https://github.com/PX4/Hardware}.

Note that the EEPROM memory chip is not currently used by the firmware,
but it is required nevertheless because future versions of the firmware may need it.

\begin{figure}[hbt]
	\centering
	\includegraphics[page=1,angle=90,origin=c,width=1\textwidth]{sapog_reference_hardware.pdf}
	\caption{Reference schematic, page 1}
\end{figure}

\begin{figure}[hbt]
	\centering
	\includegraphics[page=2,angle=90,origin=c,width=1\textwidth]{sapog_reference_hardware.pdf}
	\caption{Reference schematic, page 2}
\end{figure}
\end{document}
